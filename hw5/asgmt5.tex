\documentclass[11pt]{article}
\usepackage{amsmath,amssymb,amsthm,enumitem,clrscode3e}
\usepackage[margin=1in]{geometry}

\title{\bf Assignment 5 \\[2ex]
\rm\normalsize Due: February 25, 2019}
\date{}
\author{
  Kanyid, Bradon\\
  \texttt{bradon.kanyid@pdx.edu}
  \and
  Reimer, Daniel\\
  \texttt{daniel.reimer@pdx.edu}
}
\begin{document}
\maketitle

\begin{center}
    \fbox{\fbox{\parbox{\textwidth}{\centering
    Your solutions must be typed (preferably typeset in \LaTeX ) and submitted as a hard-copy at the beginning of
    class on the day its due.
    \\

    When asked to provide an algorithm you need to give well formatted pseudocode, a description of how your
    code solves the problem, and a brief argument of its correctness.
    }}}
\end{center}

\paragraph{Problem 1: Highway Safety [10 points]}
As is well-known, America’s highway infrastructure is crumbling. Yet travel must continue.
Suppose you are given a map of U.S. cities and roads connecting them that shows, for every road
segment, the probability of traveling down that segment {\em safely}, i.e. without destroying your axle
in a pothole, falling into a river due to a broken bridge, etc. Design and analyze an algorithm to
determine the safest route from Portland to your preferred summer vacation spot.

\paragraph{}
Stated more formally, suppose you are given a directed graph $G = (V,E)$, where every edge $e$ has an independent
safety probability $p(e)$. The safety of a path is the product of the safety probabilities
of its edges. Design and analyze an algorithm to determine the safest path from a given start vertex
$s$ to a given target vertex $t$.

\subsection*{Solution}

This problem is a modification of Dijkstra's algorithm. Instead of minimizing distance, though, we are maximizing safety probability. The rest of the algorithm is basically untouched, except we need to multiply our independent safety probabilities together to build our overall safety probability.

\subsubsection*{Analysis}

Since this is a variation on Djikstra's, it follows the complexity analysis as well. The worst-case time-complexity is $O(|V|^2)$, for our simple implementation below. There are optimizations that can be done using a Fibonacci heap or binary heap, which can reduce the complexity to $O(|E| + |V| \log |V|)$

\subsubsection*{Pseudocode}
\begin{codebox}
\li def calculateSafety(G, start, target):
\Then
  \li  Q = new Set()

  \li
  \li \For each vertex v in G:
  \Then
    \li  safety[v] = -INFINITY
    \li  prev[v] = UNDEFINED
    \li  add v to Q
  \End

  \li
  \li // the safest drive is one you never take
  \li safety[start] = 1

  \li
  \li \While Q is not empty:
  \Then
    \li u = vertex in Q with max safety[u]

    \li remove u from Q

    \li // we can end early if we've reached the target
    \li \If u == target:
    \Then
      \li return safety[], prev[]
    \End

    \li \For each neighbor v still in Q of u:
    \Then
      \li // calculate potential new path safety
      \li alt = safety[u] * p(u, v)
      \li \If alt $>$ safety[v]:
      \Then
        \li safety[v] = alt
        \li prev[v] = u
      \End
    \End
  \End

  \li return safety[], prev[]
\End
\li

\li def safestPath(G, start, target):
\Then
  \li safety, prev = calculateSafety(G, start, target)
  \li
  \li S = new Sequence()
  \li u = target
  \li \If prev[u] is defined or u == source:
  \Then
    \li // build up path from target back to start
    \li \While u is defined:
    \Then
      \li prepend S with u
      \li u = prev[u]
    \End
  \End

\end{codebox}


\paragraph{Problem 2: MST}
The {\em cut property} makes it possible to build minimum
spanning trees greedily, starting from an empty graph and adding one edge at a time. A different approach
is to start with the original graph and remove edges greedily, one at a time, until an MST remains. A
scheme of this second type can be justified by the following property.
\begin{quote}
Pick any cycle in the graph, and let $e$ be the heaviest edge in that cycle. Then there is a
minimum spanning tree that does not contain $e$.
\end{quote}
\subparagraph{(a) [5 points]}
Prove this {\em cycle property}.

\subsection*{Solution}
The linear time solution is to iterate through the array, and at each element compare the left, right, and middle element. If the middle element is the biggest, then that is the peak. This is a linear time solution because every element is only visited once and a constant time comparison is done to see if the middle is the largest.

\begin{verbatim}
  peakFinding(xs) {
    for(int i=0; i<len(xs); i++) {
      if(xs[i-1] < xs[i+1] < xs[i]) {
        return i
      }
    }
  }
\end{verbatim}


\subparagraph{(b) [5 points]}
Use the property to justify the following MST algorithm. The input is an undirected graph
$G = (V, E)$ with edge weights $\{ w_e \}$.

\begin{codebox}
\li sort the edges according to their weights
\li \For each edge $e \in E$, in decreasing order of $w_e$
    \Then
    \li \If $e$ is part of a cycle of $G$
        \Then
        \li $G = G - e$ (that is, remove $e$ from $G$)
        \End
    \End
\li \Return G
\end{codebox}

To solve this problem, the algorithm starts in the middle of the array. It then finds the biggest number between the left, right, and the middle. If the middle is the biggest, then that is the peak. Otherwise, it selects the middle towards the side of the bigger number. The process is repeated over again until peak is found.
This algorithm is O(logn) because each iteration, the possible selection of numbers is halved. Therefore logarithmic.

\begin{verbatim}
  peakFinding(xs) {
    let middle = len(xs) / 2
    let left   = 0
    let right  = len(xs)-1
    while(true) {
      if(xs[middle-1] > xs[middle+1] > xs) {
        old_middle=middle
        middle=(left + right) / 2
        right=old_middle
      }
      elseif(xs[middle+1] > xs[middle-1] > xs) {
        old_middle=middle
        middle=(left + right) / 2
        left=old_middle
      }
      else {
        return middle
      }
    }
  }
\end{verbatim}


\subparagraph{(c) [5 points]}
On each iteration, the algorithm must check whether there is a cycle containing a specific edge $e$.
Give a linear-time algorithm for this task, and justify its correctness.

The maximum value will always be a peak. In this problem, the square matrix can be represented as a one dimensional array. The data is sorted and a maximum could be retrieved in constant time. Once the maximum is found, the index can be found in constant time by mapping the one dimensional index back to two dimensional. Since the sorting algorithn used is O(nlogn) and the other operations are all in constant time, the overavll complexity is O(nlogn).

\begin{verbatim}
  2dPeakFinding(xs) {
    let m = square matrix dimension
    let sortedXs = mergeSort(xs)
    max(sortedXs)
    return (indexOf(max) / m, indexOf(max) % m)  # (x,y)
}
\end{verbatim}


\subparagraph{(d) [5 points]}
What is the overall time complexity of this algorithm, in terms of $|E|$? Explain your answer.

\subsection*{Solution}

The following algorithm is the fastest we could determine:

1) Calculate every subset of the graph
2) Remove subsets that contain restaurants on adjacent vertices
3) Compute total profit for each remaining subset
4) Take the max value from these subsets

Further analysis shows that this problem is a form of the Maximum Independent Set problem, and thus is NP-Hard. Thus, there is no known polynomial-time solution.


\end{document}
