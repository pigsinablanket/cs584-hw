\subsection*{Solution}

Vertices = locations in general, including potential locations of firehouses. Assume unselected vertices are houses being covered. Selected vertices are replaced with firehouses.

Edges = distance between vertices/locations. 

f = |F| where F is a subset of V, where firehouses will be placed.

This problem is NP-Complete because we can verify answers easily, in polynomial time. Look at each vertex in F, and mark each vertex you can get to in d distance or less. Afterwards, there should be no unmarked vertices in G. If this is the case, then you have a solution that meets the criteria.

You can reduce set-cover to firehouses problem as follows:

Converting from Firehouses to set-cover is as follows:

For every vertex v in V, we do a DFS and collect all the vertices on the way where the total edge distance < = d. Those vertices make up Si, a subset of the total set space. We know that the union of all these subsets must contain every vertex, as each subset contains it's starting vertex, from V. Each subset is "centered" around it's origin vertex and the vertices it can reach. Note that subsets definitely will overlap. We also keep a dictionary mapping each subset to its origin vertex.

With these |V| subsets, we can now put them through set-cover, with minimum size subset |C| = f. If we find a solution that fits these requirements, we can get the original verteces back by using the constructed dictionary, thus leaving us with the locations of the firehouses.
