\documentclass[11pt]{article}
\usepackage{amsmath,amssymb,amsthm,enumitem,clrscode3e}
\usepackage[margin=1in]{geometry}

\title{\bf Assignment 6 \\[2ex]
\rm\normalsize Due: March 3, 2020}
\date{}
\author{
  Kanyid, Bradon\\
  \texttt{bradon.kanyid@pdx.edu}
  \and
  Reimer, Daniel\\
  \texttt{daniel.reimer@pdx.edu}
}
\begin{document}
\maketitle

\begin{center}
    \fbox{\fbox{\parbox{\textwidth}{\centering
    Your solutions must be typed (preferably typeset in \LaTeX ) and submitted as a hard-copy at the beginning of
    class on the day its due.
    \\

    When asked to provide an algorithm you need to give well formatted pseudocode, a description of how your
    code solves the problem, and a brief argument of its correctness.
    }}}
\end{center}

\paragraph{Problem 1: Longest-Probe Bound for Hashing}
Suppose we use an open-addressed hash table (section 11.4 in CLRS) of size $m$ to store $n \leq \frac m2$ items.

\subparagraph{(a) [10 points]}
Assuming simple uniform hashing, show that for $i = 1,2,...,n$, the probability is at most $2^{-k}$ that the $i$th
insertion requires strictly more than $k$ probes.

\subsection*{Solution}

Under the assumption of uniform hashing, the likelihood for any insertion to have a colision is $n/m$ where $n$ is the number of probes is $m$ is the size. We know that our open-addressed table has $n < m/2$, so the load factor is known to be strictly $< 1/2$.

The probability of needing more than $k$ probes will be some value $(1/2)*(1/2)*...*(1/2)$ for $k$ times, or $(1/2)^k$. So $(n/m)^k < (1/2) ^k = 2^{-k}$.


\subparagraph{(b) [10 points]}
Assuming simple uniform hashing, show that for $i = 1,2,...,n$, the probability is $O(\frac 1{n^2})$ that the
$i$th insertion requires more than $2 \lg n$ probes.

\subsection*{Solution}

Under the assumption of uniform hashing, the likelihood for any insertion to have a colision is $n/m$ where $n$ is the number of probes is $m$ is the size. We know that our open-addressed table has $n < m/2$, so the load factor is known to be strictly $< 1/2$.

\begin{align*}
  2 lg n &== (1/2)^{(2 lg n)} \\
         &== 2^{(-2 lg n)} \\
         &== 1/{(2^{lg n})^2} \\
         &== 1/n^2 \\
\end{align*}

Therefore the probability of needing more than $2lg(n)$ probes will be $1/n^2$.


\paragraph{Problem 2: Building a Queue using Stacks}
It is possible to build a {\em queue} (FIFO) using two stacks. Assume that the stacks have three operations, {\em push}, {\em pop}, and {\em isEmpty}, each with cost 1. A queue can be implemented as follows:
\begin{itemize}
\item {\em enqueue:} push item $x$ onto stack 1
\item {\em dequeue:} if stack 2 is empty then pop the entire contents of stack 1 pushing each element in turn onto stack 2. Now pop from stack 2 and return the result.
\end{itemize}
A conventional worst-case analysis would establish that {\em dequeue} takes $O(n)$ time, but this is clearly a weak bound for a sequence of operations, because very few dequeues will actually take that long. To simplify your analysis only consider the cost of the push and pop operations.

\subparagraph{(a) [10 points]}
Using the aggregate method show that the amortized cost of each {\em enqueue} and {\em dequeue} is constant.

\subsection*{Solution}

Even though a single dequeue operation can be quite expensive (in the case where many enqueues have happened in a row without a dequeue), any sequence of n enqueue and dequeue operations on an initially empty pair of stacks can cost at most $O(n)$. 

Enqueue is constant work, always pushing one item onto the first stack.
Dequeue has two paths:
\begin{itemize}
\item If stack 2 is non-empty, the work is constant and pops one element from the second stack.
\item If stack 2 is empty, there are n pops and pushs from stack 1 to stack 2, where n is the number of elements on stack 1 prior to the operation.
\end{itemize}

We can dequeue each object from the stack at most once for each time we have enqueued it onto the stack. Therefore, the number of times that dequeue can be called on a nonempty stack is at most the number of enqueue operations, which is at most n. For any value of n, any sequence of n enqueue and dequeue operations takes a total of $O(n)$ time. The average cost of an operation is $O(n)/n == O(1)$.


\subparagraph{(b) [10 points]}
Using the accounting method show that the amortized cost of each {\em enqueue} and {\em dequeue} is constant.

\subsection*{Solution}

We are given the actual costs of the three operations, push, pop, and isEmpty, each with cost 1. We will define our own amortized costs for the larger operations: 
\begin{itemize}
\item {\em Enqueue:} amortized cost of 4
\item {\em Dequeue:} amortized cost of 1
\end{itemize}

As in 2a, there's 3 meta-operations that can happen:
\begin{itemize}
\item {\em Enqueue} 
\item {\em Dequeue (Empty Stack 2)} 
\item {\em Dequeue (Non-Empty Stack 2)} 
\end{itemize}

Let's break down the cost of each operation.

\subsubsection*{\em Enqueue:} 
\begin{itemize}
\item 1 push to stack 1
\end{itemize}

An enqueue uses 1 credit of the 4 credits it is given, leaving 3 credits remaining.

\subsubsection*{\em Dequeue (Empty Stack 2):}
\begin{itemize}
\item 1 isEmpty (paid by the item being dequeued)
\item 1 pop from stack 1 (paid by every item on stack 1)
\item 1 push to stack 2 (paid by every item on stack 1)
\item 1 pop from stack 2 (paid by the item being dequeued)
\end{itemize}

The cost per item is 2 for every item moving from stack 1 to stack 2, leaving a credit of 2 for those items. The item destined to be dequeued has also paid for the isEmpty operation, as well as the cost of the pop from stack 2. The final result is that stack 1 is empty, and the remaining items on stack 2 have a credit of 2, which brings us to the last case...

\subsubsection*{\em Dequeue (Non-Empty Stack 2):}
\begin{itemize}
\item 1 isEmpty 
\item 1 pop from stack 2
\end{itemize}

A non-empty stack 2 will quickly remove the top item from stack 2, and that item pays its two remainig credits for the two operations.


\subparagraph{(c) [10 points]}
Using the potential method show that the amortized cost of each {\em enqueue} and {\em dequeue} is constant.

\subsection*{Solution}

For any item in the first stack, there's a pop, push, and pop left for it
for any item in the second stack, there's just a pop left.
so if we have j items in stack 1 and k items in stack 2, a reasonable potential would be 4j + 2k

enqueueing thus increases the potential by 4
dequeueing if isEmpty is false is -2
dequeueing if isEmpty is true is -4j-2

So the basic idea for amortized analysis now that we have a potential function is that the amortized cost of an operation is cost of operation + potential after operation - potential before operation
Which is just a theorem you get to use
the cost of a dequeue if isEmpty is false is 1 for isEmpty and 1 for pop, so 2
cost of operation + potential after operation - potential before operation == cost of operation + delta in potential
and the change in potential when isEmpty is false is -2, as we established earlier

2 + -2 = 0, so the amortized cost of a dequeue is nothing if isEmpty is false

if isEmpty is true, then the cost is 1 + 2j + 1 
2j because j pops from stack 1, j pushes to stack 2
isempty + j pop/pushes + last pop

and that reduces the potential by 4j from stack one, but increases it by 2j in stack 2
but then decreases it by 2 with the removal of the item
which is a net delta in potential of -2j-2
2j+2-2j-2 = 0
So the amortized cost in that case is also 0

enqueue costs 1 push to perform
and increases potential by 4
and thus has an amortized cost of 5
which is constant
0 is O(1), 5 is O(1), everything here is amortized O(1)



\end{document}
