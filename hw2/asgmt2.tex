\documentclass[11pt]{article}
\usepackage{amsmath,amssymb,amsthm,enumitem,clrscode3e}
\usepackage[margin=1in]{geometry}

\title{\bf Assignment 2 \\[2ex]
\rm\normalsize Due: January 23, 2020}
\date{}
\author{
  Kanyid, Bradon\\
  \texttt{bradon.kanyid@pdx.edu}
  \and
  Reimer, Daniel\\
  \texttt{daniel.reimer@pdx.edu}
}
\begin{document}
\maketitle

\paragraph{Problem 1: Jarvis March (Gift Wrapping Algorithm)}
For this question you will need to research the {\em Jarvis March} convex hull algorithm. Be sure to cite your sources (ACM or IEEE formatted is preferred).
\subparagraph{(a) [10 points]}
Give pseudocode describing the Jarvis March algorithm, a brief description of how it works, and explain its best and worst case efficiency.
\subsection*{Solution}
Another way of stating $\log_k n \in O(\lg n)$ for any $k > 1$ is:
\begin{align*}
\log_k n &\leq \lg n \text{for any}  k > 1 \\
\frac{\lg n}{\lg k} &\leq \frac{\lg n}{\lg 2}
\end{align*}
Since $\lg 2 = 1$, \\
\begin{align*}
\frac{\lg n}{\lg k} &\leq \lg n \\ 
\left(\frac{1}{\lg k}\right)\lg n &\leq \lg n \\
\left(\frac{1}{\lg k}\right) &\leq 1 \text{for any} k > 1 \\
\end{align*}

\subparagraph{(b) [5] points}
Give an example input on which Jarvis March will perform significantly better than Graham's scan and explain why it will perform better.
\subsection*{Solution}
\begin{align*}
    T(n) &= 2T\left( \frac n2 \right) + n \lg n \\
    T(\frac{n}{2}) &= 2[2T\left( \frac{n}{4} \right) + \frac{n}{2} \lg \frac{n}{2}] + n \lg n \\
    T(n) &= 2^{k}T\left( \frac{n}{2^{k}} \right) + k \left(n \lg n\right) \\
    \text{Assume that:} \\
    T\left(\frac{n}{k}\right) &= T\left(1\right) \\
    \text{That implies:} \\
    \frac{n}{2^{k}} &= 1 \\
    n &= 2^{k} \\
    k &= \lg n \\
    \text{Substituting } k = \lg n \\
    T(n) &= 2^{\lg n}T\left( \frac{n}{2^{\lg n}} \right) + \left(\lg n\right) \left(n \lg n\right) \\
    \text{Simplifying...} \\
    T(n) &= nT\left(1\right) + n \lg^{2} n \\
         &= O(n \lg^{2} n)
\end{align*}

\subparagraph{(c) [5] points}
Give an example input on which Graham's Scan will perform significantly better than Jarvis March and explain why it will perform better.
\subsection*{Solution}
Similar to Problem 1a, another way of stating $f(n) \in O(g(n))$ is: 
\begin{align*}
  f(n) &\le g(n) \\
  \intertext{To prove that $2^{f(n)} \in O(2^{g(n)})$, then, one must only state in the same way that:} \\
  2^{f(n)} &\le 2^{g(n)} \\
  \intertext{Taking the lg of each side, the inequality holds,} \\
  \lg\left(2^{f(n)}\right) &\le \lg\left(2^{g(n)}\right) \\
  f(n) &\le g(n) \\
  \intertext{Thus,} \\
  2^{f(n)} &\in O(2^{g(n)}) \\
\end{align*}



\paragraph{Problem 2: Find the Missing Number}
You are given a list of $n-1$ integers $A$, in the range of 1 to $n$. There are no duplicates in the list. One of the integers is missing.
(Feel free to assume that $n = 2^m$ for some integer $m$)
\subparagraph{(a) [5 points]}
Give an efficient algorithm for finding the missing number, show its complexity, and argue its correctness.
(You should try for $O(n)$-time and $O(1)$-space, less efficient solutions will still get partial credit)
\subsection*{Solution}
The linear time solution is to iterate through the array, and at each element compare the left, right, and middle element. If the middle element is the biggest, then that is the peak. This is a linear time solution because every element is only visited once and a constant time comparison is done to see if the middle is the largest.

\begin{verbatim}
  peakFinding(xs) {
    for(int i=0; i<len(xs); i++) {
      if(xs[i-1] < xs[i+1] < xs[i]) {
        return i
      }
    }
  }
\end{verbatim}

\subparagraph{(b) [10 points]}
For this question you are not allowed to access an entire integer with a single operation. The elements of the list are represented in binary, and the only operation you can use to access them is \proc{GetBinaryDigit(A[i],j)} which returns the $j$th bit of element $A[i]$ which runs in constant time. Give an efficient algorithm for finding the missing number under these constraints, show its complexity, and argue its correctness.
(You should try for $O(n)$-time and $O(log n)$-space, less efficient solutions will still get partial credit)

Example: If we run \proc{GetBinaryDigit($A[i]$,$j$)} with $A[i]=29$ and $j=2$, it would return a $0$ since $29=11101$.
To solve this problem, the algorithm starts in the middle of the array. It then finds the biggest number between the left, right, and the middle. If the middle is the biggest, then that is the peak. Otherwise, it selects the middle towards the side of the bigger number. The process is repeated over again until peak is found.
This algorithm is O(logn) because each iteration, the possible selection of numbers is halved. Therefore logarithmic.

\begin{verbatim}
  peakFinding(xs) {
    let middle = len(xs) / 2
    let left   = 0
    let right  = len(xs)-1
    while(true) {
      if(xs[middle-1] > xs[middle+1] > xs) {
        old_middle=middle
        middle=(left + right) / 2
        right=old_middle
      }
      elseif(xs[middle+1] > xs[middle-1] > xs) {
        old_middle=middle
        middle=(left + right) / 2
        left=old_middle
      }
      else {
        return middle
      }
    }
  }
\end{verbatim}


\end{document}
