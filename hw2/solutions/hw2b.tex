\subsection*{Solution}
\subsubsection*{Pseudocode}
\begin{verbatim}
  findMissingNumberB(A) {
    // pre-compute some useful re-used values
    aLen = len(A)
    lgA = lg(aLen)

    // create array to hold intermediate results
    prevResult = new Array(lgA)

    // loop through every element of A
    for(int i=0; i < aLen; i++) {
      // check each bit for each element
      for(int j=0; j < lgA; j++) {
        // xor this element's bit w/ a running 'sum' of the previous xor'ed bits
        prevResult[j] ^= GetBinaryDigit(A[i], j)
      }
    }

    // now combine the stored bits back together into a result
    result=0;
    for (int j=lgA; j > 0; j--) {
      // build the resultant integer from it's bits
      result <<= prevResult[j];
    }
    // ... and return it
    return result;
  }
\end{verbatim}

\subsubsection*{Complexity}
Like in part 2a, this algorithm has a single walk of the elements of A, and it doesn't matter that we also walk across the bits of each element, so it's complexity is $O(n)$-time. We pre-allocate our result storage as $\lg(n) $elements, so it has complexity of $O(\lg n)$-space.

\subsubsection*{Correctness}
Because we're assuming that $n=2^{m}$, we can make further guarantees of what a full set of numbers from $1..n$ would look like, represented in binary. We know that any number in that range can be represented in $\lg(n)$ bits, and that \underline{every} bit-pattern should be represented within those $\lg(n)$ bits.

For every binary digit, there should be an even number of occurances of 1's in a fully-represented bit-pattern that spans the space. Using xor allows us to track the even/odd status of each binary digit. In a fully-represented bit-space made up of $n$ bits, xor'ing every number with the other will leave you with all zero bits in your reduction. In our array $A$, the only digits that won't have 0's in their position are digits that were not fully represented in the bit-pattern, i.e. were missing from the array.

To show this further, if someone were to xor the found missing number with the reduction variable, every digit would be 0, as at that point all bit-patterns within those $\lg(n)$ bits will have been represented.

It's important to leave the caveat that this is only true if $n=2^{n}$ and we are working with binary digits.
