\documentclass[11pt]{article}
\usepackage{amsmath,amssymb,amsthm,enumitem}
\usepackage[margin=1in]{geometry}

\title{\bf Assignment 1 \\[2ex]
\rm\normalsize Due: January 16, 2020}
\date{}
\author{}

\begin{document}
\maketitle

\begin{center}
    \fbox{\fbox{\parbox{5.5in}{\centering
        Your solutions must be typed (preferably typeset in \LaTeX ) and submitted as a hard-copy at the beginning of
        class on the day its due.
        \\

        When asked to provide an algorithm you need to give well formatted pseudocode, a description of how your
        code solves the problem, and a brief argument of its correctness.
    }}}
\end{center}

\paragraph{Problem 1: Asymptotic Analysis Practice}
\subparagraph{(a) [5 points]}
Prove or disprove that $\log_k n \in O(\lg n)$ for any $k > 1$. (Note that $\lg$ refers to $\log_2$)
\\
\\
Another way of stating $\log_k n \in O(\lg n)$ for any $k > 1$ is:
\begin{align*}
\log_k n &\leq \lg n \text{ for any }  k > 1 \\
\frac{\lg n}{\lg k} &\leq \frac{\lg n}{\lg 2}
\end{align*}
Since $\lg 2 = 1$, \\
\begin{align*}
\frac{\lg n}{\lg k} &\leq \lg n \\
\left(\frac{1}{\lg k}\right)\lg n &\leq \lg n \\
\left(\frac{1}{\lg k}\right) &\leq 1 \text{ for any } k > 1 \\
\end{align*}


\subparagraph{(b) [5 points]}
The following recurrence relation solves to $O(n \lg^2 n)$. Prove this by substition. Do not use the Master method.
\begin{align*}
T(n) &= 2T\left( \frac n2 \right) + n \lg n \\
T(1) &= 0
\end{align*}
\subparagraph{(c) [5 points]}
Suppose that $f(n)$ and $g(n)$ are non-negative functions. Prove or disprove the following: if $f(n) \in O(g(n))$ then $2^{f(n)} \in O(2^{g(n)})$.


\paragraph{Problem 2: Peak-finding}
Given a set of real numbers stored in an array $A$ find the index of a {\em Peak}, where a {\em Peak} is defined as an
element that is larger or equal to the both the elements on its sides. (Note: you only need to return {\em a} peak, not
the highest one.)\\

Example array:
                {\tt
                \begin{tabular}{|*{9}{r|}}
                    \hline
                    -2 & \ 6 & -1 & \ 4 & \ 9 & -5 & \ 5  \\
                    \hline
                \end{tabular}
                }

Assuming the array one based it has peaks at indices $\{2,5,7 \}$.

\subparagraph{(a) [5 points]}
Give a linear time algorithm to solve this problem. (This should be obvious)
\subparagraph{(b) [10 points]}
Give a $O(\log n)$ time algorithm to solve this problem.

To solve this problem, the algorithm starts in the middle of the array. It then finds the biggest number between the left, right, and the middle. If the middle is the biggest, then that is the peak. Otherwise, it selects the middle towards the side of the bigger number. The process is repeated over again until peak is found.
This algorithm is O(logn) because each iteration, the possible selection of numbers is halved. Therefore logarithmic.

\begin{verbatim}
  peakFinding(xs) {
    let middle = len(xs) / 2
    let left   = 0
    let right  = len(xs)-1
    while(true) {
      if(xs[middle-1] > xs[middle+1] > xs) {
        old_middle=middle
        middle=(left + right) / 2
        right=old_middle
      }
      elseif(xs[middle+1] > xs[middle-1] > xs) {
        old_middle=middle
        middle=(left + right) / 2
        left=old_middle
      }
      else {
        return middle
      }
    }
  }
\end{verbatim}


\subparagraph{(c) [10 points]}
What if instead of a simple array you are given a square matrix, where a {\em Peak} is now defined as an element larger
or equal to its four neighbours. Give a $O(n \log n)$ solution to this variant of the problem.

\input{solutions/hw12c2c.tex}

\end{document}
