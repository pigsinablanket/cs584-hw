\documentclass[11pt]{article}
\usepackage{amsmath,amssymb,amsthm,enumitem,clrscode3e}
\usepackage[margin=1in]{geometry}

\title{\bf Proposal \\[2ex]
\rm\normalsize Due: January 30, 2020}
\date{}
\author{
  Kanyid, Bradon\\
  \texttt{bradon.kanyid@pdx.edu}
  \and
  Reimer, Daniel\\
  \texttt{daniel.reimer@pdx.edu}
}
\begin{document}
\maketitle


The problem we intend to focus on is to compare treaps and skip lists to some deterministic self-balancing trees and determine how close to balanced they are as well as how well they perform. In other words, we will be comparing trees that balance through randomness like the treap and skip list algorithms to a tree that maintains a balanced tree programatically.

The algorithms we are planning to implement is treaps, skip lists, and a red-black tree which is a deterministic self-balancing tree whereas treaps and skip lists are probalistic and random. There are deterministic ways to implement both treaps and skip lists but for the sake of the problem we are focusing on, these two algorithms will be implemented through random and probablistic means.

The programming language to implement these algorithms will be Python. The reason behind our choice is because of minimal development time, is known between both of us, and rich library to aid with graph generation. Another benefit to using python is that the implementation of python for complexity in both space and time is easy to reason about compared to some other languages.

Bullet point 4: A rough outline of what you’re planning to do to compare those algorithms.

\end{document}
