\documentclass[11pt]{article}
\usepackage{amsmath,amssymb,amsthm,enumitem,clrscode3e}
%\usepackage[margin=1in]{geometry}

\title{\bf Assignment 4 \\[2ex]
\rm\normalsize Due: February 18, 2020}
\date{}
\author{
  Kanyid, Bradon\\
  \texttt{bradon.kanyid@pdx.edu}
  \and
  Reimer, Daniel\\
  \texttt{daniel.reimer@pdx.edu}
}

\begin{document}
\maketitle

\begin{center}
    \fbox{\fbox{\parbox{\textwidth}{\centering
    Your solutions must be typed (preferably typeset in \LaTeX ) and submitted as a hard-copy at the beginning of
    class on the day its due.
    \\

    When asked to provide an algorithm you need to give well formatted pseudocode, a description of how your code solves the problem, and a brief argument of its correctness.
    }}}
\end{center}

\paragraph{Problem 1: Maximum Subarray Sum}
The Maximum Subarray Sum problem is the task of finding the contiguous subarray with largest sum in a given array of integers.
Each number in the array could be positive, negative, or zero. For example: Given the array $[-2, 1, -3, 4, -1, 2, 1, -5, 4]$
the solution would be $[4, -1, 2, 1]$ with a sum of 6.

\subparagraph{(a) [5 points]}
Give a brute force solution for this problem with complexity of $O(n^2)$.
Another way of stating $\log_k n \in O(\lg n)$ for any $k > 1$ is:
\begin{align*}
\log_k n &\leq \lg n \text{for any}  k > 1 \\
\frac{\lg n}{\lg k} &\leq \frac{\lg n}{\lg 2}
\end{align*}
Since $\lg 2 = 1$, \\
\begin{align*}
\frac{\lg n}{\lg k} &\leq \lg n \\ 
\left(\frac{1}{\lg k}\right)\lg n &\leq \lg n \\
\left(\frac{1}{\lg k}\right) &\leq 1 \text{for any} k > 1 \\
\end{align*}


\subparagraph{(b) [10 points]}
Give a divide and conquer solution for this problem with complexity $O(n \log n)$.
\subsection*{Solution}
\begin{align*}
    T(n) &= 2T\left( \frac n2 \right) + n \lg n \\
    T(\frac{n}{2}) &= 2[2T\left( \frac{n}{4} \right) + \frac{n}{2} \lg \frac{n}{2}] + n \lg n \\
    T(n) &= 2^{k}T\left( \frac{n}{2^{k}} \right) + k \left(n \lg n\right) \\
    \text{Assume that:} \\
    T\left(\frac{n}{k}\right) &= T\left(1\right) \\
    \text{That implies:} \\
    \frac{n}{2^{k}} &= 1 \\
    n &= 2^{k} \\
    k &= \lg n \\
    \text{Substituting } k = \lg n \\
    T(n) &= 2^{\lg n}T\left( \frac{n}{2^{\lg n}} \right) + \left(\lg n\right) \left(n \lg n\right) \\
    \text{Simplifying...} \\
    T(n) &= nT\left(1\right) + n \lg^{2} n \\
         &= O(n \lg^{2} n)
\end{align*}


\subparagraph{(c) [10 points]}
Give a dynamic programming solution for this problem with complexity $O(n)$.
\subsection*{Solution}
Graham's Scan will perform significantly better than Jarvis March when all the points are on the convex hull. When all the points are on the hull for Jarvis March, every point in the set must be compared with every point on the hull and when every point is on the hull, the time complexity is $O(hn) == O(n^2)$ when $h=n$, where h is number of points on the hull. For Graham's scan, the complexity of the sorting algorithm used is the complexity of Graham's scan therefore will always be $O(n \log n)$ , no matter the size of the input.


\paragraph{Problem 2: Restaurant Placement}
A new restaurant chain is opening and you have been given the task of selecting the restaurant locations with the
goal of maximizing their total profit.\\

The street network is described as an undirected graph $G=(V,E)$, where the
potential restaurant sites are the vertices of the graph. Each vertex $v$ has a non-negative integer value $p_v$,
which describes the potential {\em profit} of site $v$. Two restaurants cannot be built on adjacent vertices.
You are supposed to design an algorithm that outputs the chosen set $U \subseteq V$ of sites that maximizes the total
profit $\sum_{v \in V} p_v$.\\

For parts (a)-(c), suppose that the street network $G$ is {\em acyclic}, i.e. a tree.

\subparagraph{(a) [5 points]}
Consider the following {\em greedy} restaurant placement algorithm. Choose the highest profit vertex $v_0$ in the
tree, breaking ties according to some ordering on vertex names, and put it into $U$. Remove $v_0$ and all of its
neighbors from $G$. Repeat until no vertices remain. Give a counterexample to show that this algorithm does not
always give a restaurant placement with maximal profit.
\subsection*{Solution}
The linear time solution is to iterate through the array, and at each element compare the left, right, and middle element. If the middle element is the biggest, then that is the peak. This is a linear time solution because every element is only visited once and a constant time comparison is done to see if the middle is the largest.

\begin{verbatim}
  peakFinding(xs) {
    for(int i=0; i<len(xs); i++) {
      if(xs[i-1] < xs[i+1] < xs[i]) {
        return i
      }
    }
  }
\end{verbatim}


\subparagraph{(b) [10 points]}
Give an efficient algorithm to determine a placement with maximum profit.
\subsection*{Solution}
To solve this problem, the algorithm starts in the middle of the array. It then finds the biggest number between the left, right, and the middle. If the middle is the biggest, then that is the peak. Otherwise, it selects the middle towards the side of the bigger number. The process is repeated over again until peak is found.
This algorithm is $O(\log n)$ because each iteration, the possible selection of numbers is halved. Therefore logarithmic.

\begin{verbatim}
  peakFinding(xs) {
    let middle = len(xs) / 2
    let left   = 0
    let right  = len(xs)-1
    while(true) {
      if(xs[middle-1] > xs[middle+1] > xs) {
        old_middle=middle
        middle=(left + right) / 2
        right=old_middle
      }
      elseif(xs[middle+1] > xs[middle-1] > xs) {
        old_middle=middle
        middle=(left + right) / 2
        left=old_middle
      }
      else {
        return middle
      }
    }
  }
\end{verbatim}


\subparagraph{(c) [5 points]}
Suppose that, in the absence of good data, the restaurant chain decides that all sites are equally good. The goal
therefore is to simply find the placement with the maximum number of locations. Give a simple greedy algorithm for
this case and argue its correctness.
\subsection*{Solution}
The maximum value will always be a peak. In this problem, the square matrix can be represented as a one dimensional array. The data is sorted and a maximum could be retrieved in constant time. Once the maximum is found, the index can be found in constant time by mapping the one-dimensional index back to two-dimensional. Since the sorting algorithm used is $O(n \log n)$ and the other operations are all in constant time, the overall complexity is $O(n \log n)$.

\begin{verbatim}
  2dPeakFinding(xs) {
    let m = square matrix dimension
    let sortedXs = mergeSort(xs)
    let max = max(sortedXs)
    return (indexOf(max) / m, indexOf(max) % m)  # (x,y)
}
\end{verbatim}


\subparagraph{(d) [5 points]}
Suppose that the graph is arbitrary and not necessarily {\em acyclic}. Give the fastest correct algorithm you can
for solving the problem.
\subsection*{Solution}

line 1: E lg E
line 2: E
line 3: E in an inner loop (calling cycle checker for e)
lines 2-4: E^2


\end{document}
