\subsection*{Solution}

Split array into two subarrays
Recurse on left and right half, returning the sum of the left, right, and the combination of the two

\subsubsection*{Pseudocode}
\begin{verbatim}
max_interval (xs, l_index, r_index):
  if l_index = r_index:
    return xs[l_index]

  midpoint = (r_index + l_index) / 2
  l_max = max_interval(xs[l_index .. midpoint], l_index, midpoint)
  r_max = max_interval(xs[midpoint .. r_index], midpoint, r_index)

  curr_sum = 0
  left_sum = - inf
  for i in xs[midpoint .. l_index]:
    curr_sum += xs[i]
    if curr_sum > left_sum:
      left_sum = curr_sum

  curr_sum = 0
  right_sum = - inf
  for i in xs[midpoint .. r_index]:
    curr_sum += xs[i]
    if curr_sum > right_sum:
      right_sum = curr_sum

  mid_max = left_sum + right_sum

  return max(l_max, r_max, mid_max)
\end{verbatim}

\subsubsection*{Description}
This code works by splitting the array into two subarrays and recursing down the left and right half and finding the max of the subarrays under each branch. Following, find the max of the subarrays that overlap each of the left and right branches. The max of the left branch, right branch, and the max between the two branches is then returned.

\subsubsection*{Efficiency}
The overall effeciency is O(nlogn) since to calculate the left and right branches is overall O(nlogn) since the operation is essientally depth first search. Calculating the midpoint is done in two steps each only traversing over half the list each and therefore O(n) for that step. Overall the efficiency is O(nlogn).
