\documentclass[11pt]{article}
\usepackage{amsmath,amssymb,amsthm,enumitem,clrscode3e}
%\usepackage[margin=1in]{geometry}

\title{\bf Assignment 3 \\[2ex]
\rm\normalsize Due: February 4, 2020}
\date{}
\author{
  Kanyid, Bradon\\
  \texttt{bradon.kanyid@pdx.edu}
  \and
  Reimer, Daniel\\
  \texttt{daniel.reimer@pdx.edu}
}

\begin{document}
\maketitle

\paragraph{Problem 1: Lower Bounds [10 points]}
Prove that any comparison-based algorithm for constructing a binary search tree from an arbitrary list of $n$ elements takes $\Omega(n \log n)$ time in the worst case. (Hint: Consider how to reduce the sorting problem to performing a set of operations on a binary search tree. In other words, show that if a faster algorithm existed for constructing a binary search tree then you would violate the $\Omega(n \log n)$ comparison-based sorting lower bound.)
\subsection*{Solution}

This problem is a modification of Dijkstra's algorithm. Instead of minimizing distance, though, we are maximizing safety probability. The rest of the algorithm is basically untouched, except we need to multiply our independent safety probabilities together to build our overall safety probability.

\subsubsection*{Analysis}

Since this is a variation on Djikstra's, it follows the complexity analysis as well. The worst-case time-complexity is $O(|V|^2)$, for our simple implementation below. There are optimizations that can be done using a Fibonacci heap or binary heap, which can reduce the complexity to $O(|E| + |V| \log |V|)$

\subsubsection*{Pseudocode}
\begin{codebox}
\li def calculateSafety(G, start, target):
\Then
  \li  Q = new Set()

  \li
  \li \For each vertex v in G:
  \Then
    \li  safety[v] = -INFINITY
    \li  prev[v] = UNDEFINED
    \li  add v to Q
  \End

  \li
  \li // the safest drive is one you never take
  \li safety[start] = 1

  \li
  \li \While Q is not empty:
  \Then
    \li u = vertex in Q with max safety[u]

    \li remove u from Q

    \li // we can end early if we've reached the target
    \li \If u == target:
    \Then
      \li return safety[], prev[]
    \End

    \li \For each neighbor v still in Q of u:
    \Then
      \li // calculate potential new path safety
      \li alt = safety[u] * p(u, v)
      \li \If alt $>$ safety[v]:
      \Then
        \li safety[v] = alt
        \li prev[v] = u
      \End
    \End
  \End

  \li return safety[], prev[]
\End
\li

\li def safestPath(G, start, target):
\Then
  \li safety, prev = calculateSafety(G, start, target)
  \li
  \li S = new Sequence()
  \li u = target
  \li \If prev[u] is defined or u == source:
  \Then
    \li // build up path from target back to start
    \li \While u is defined:
    \Then
      \li prepend S with u
      \li u = prev[u]
    \End
  \End

\end{codebox}


\paragraph{Problem 2: Median of Medians}
The 'Median-of-medians' selection algorithm presented in class divides the input into groups of 5. Using a group of odd size helps keep things a little simpler (because otherwise the group medians are messier to define), but why the choice of 5?
\subparagraph{(a) [10 points]}
Show that the same argument for linear worst-case time complexity works if we use groups of size 7 instead.
\subsection*{Solution}
The linear time solution is to iterate through the array, and at each element compare the left, right, and middle element. If the middle element is the biggest, then that is the peak. This is a linear time solution because every element is only visited once and a constant time comparison is done to see if the middle is the largest.

\begin{verbatim}
  peakFinding(xs) {
    for(int i=0; i<len(xs); i++) {
      if(xs[i-1] < xs[i+1] < xs[i]) {
        return i
      }
    }
  }
\end{verbatim}


\subparagraph{(b) [10 points]}
Show that groups of size 3 results in superlinear time complexity.
To solve this problem, the algorithm starts in the middle of the array. It then finds the biggest number between the left, right, and the middle. If the middle is the biggest, then that is the peak. Otherwise, it selects the middle towards the side of the bigger number. The process is repeated over again until peak is found.
This algorithm is O(logn) because each iteration, the possible selection of numbers is halved. Therefore logarithmic.

\begin{verbatim}
  peakFinding(xs) {
    let middle = len(xs) / 2
    let left   = 0
    let right  = len(xs)-1
    while(true) {
      if(xs[middle-1] > xs[middle+1] > xs) {
        old_middle=middle
        middle=(left + right) / 2
        right=old_middle
      }
      elseif(xs[middle+1] > xs[middle-1] > xs) {
        old_middle=middle
        middle=(left + right) / 2
        left=old_middle
      }
      else {
        return middle
      }
    }
  }
\end{verbatim}


\end{document}
